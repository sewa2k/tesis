% Autogenerated translation of discusiones.md by Texpad
% To stop this file being overwritten during the typeset process, please move or remove this header

\documentclass[a4paper, 12pt]{book}
\usepackage{graphicx}
\usepackage[utf8]{inputenc}
\usepackage[a4paper]{geometry}
\usepackage{hyperref}
\pagestyle{plain}
\begin{document}

chapter\{Discusiones\}

La perdida del equilibrio Ambiente\$Leftrightarrow\$Patógeno\$Leftrightarrow\$Hospedero es la causa de la mayoria de las enfermedades presentes en la acuicultura, y es por eso que es estrictamente necesario cimentar las bases de una comprensión íntegra del sistema inmune, para así, poder generar tecnología que pueda sobreponerse a estos paradigmas. Esto tiene suma importancia sobretodo en la industria acuícola, la cual produce anualmente, y con un crecimiento constante, 148 millones de toneladas de pescado [@FAO2012], las cuales se traducen aproximadamente en 217.500 millones de dolares (USD), mas aún, de toda esa producción, 128 millones de toneladas fueron exclusivamente destinados a consumo humano, por lo que las perdidas por un brote de alguna enfermedad ascienden a millones de dolares, brotes que amenazan las año a año las operaciones acuicolas al rededor del mundo.

section\{Objetivo 1\}
Los inmunoestimulantes han surgido como una opción viable, escalable y económica para solventar parte de los problemas de la acuicultura, fortaleciendo su capacidad de respuesta inmune en distintos estadíos de desarrollo. Dentro de las formas en las cuales se pueden desplegar estos inmunoestimulantes podemos encontrar vacunas, suspensión oral y liberación en el alimento, entre otras.

Para este proyecto se describió una dieta en base a emph\{pellets\} de harina de pescado, los cuales contenían como inmunoestimulante el \$beta\$-glucano emph\{Zimosán A\}, proveniente de la levadura emph\{Saccharomyces cerevisiae\}, en una razón del 0,3\%.

Los peces se mantuvieron el CIAC alimentandose en dos grupos, control y tratado, el totalidad de la mortalidad fue del 100\%, ya que todos los peces se sacrificaron en los días determinado para ese fin dentro de la linea de tiempo del ensayo. 

Previo al desangramiento de cada pez se obtuvo el peso de cada organismo, con el fin de confeccionar un grafico para evaluar si la diferentes dietas generaban algun cambio en la masa del pez (Figura ref\{fig:pesos\}). Esto no fue así ya que el peso de ambos grupos se mantuvo constante, los datos al ser correlacionados mediante el coeficiente de Pearson obtuvieron un R = 0,962, lo que indica una correlación positiva de casi un 100\%.

section\{Objetivo 2\}
La extracción de RNA y su posterior cuantificación se mantuvo constante (Tabla ref\{tablaRNA\}), demostrando que en la mayoría de los casos se trabajó prolijamente y sin mucha diferencia entre los distintos dias de muestreo. Los dos unicos casos en que se obtuvo una concentración muy baja fue en las muestras B2 y B41, con 63,3 y 40,20 si\{nanogramo\}/si\{microlitro\} respectivamente. Esto puede haberse debido a una mala manipulacion del mortero, el cual alcanzaba temperaturas cercanas a los -150ºC al mantenerse constantemente con nitrógeno liquido,por lo tanto cuando se tomaba tejido pulverizado para agregar al tubo de homogeneización se podría haber perdido algo de muestra.

Con el RNA extraído y cuantificado se procedió a sintetizar su DNA complementario, al haberse hecho esto con Kit y Termociclador salió todo bien sin ninguna complicación, con lo que finalmente se pudo empezar a realizar la estandarización de partidores.

El primer par de partidores en estandarizar fue el de referncia, el cual tuvo una eficiencia de 96,9\% y solo un peak en la curva de disociación lo que nos corrobora que el primer, a 58ºC como temperatura de annealing, genera un solo producto y que cada ciclo dobla su cantidad inicial de templado (Figura ref\{fig:ef1a\}). Para las demas parejas de partidores correspondientes a los genes en estudio se observó la misma tendencia, generandose eficiencias de 105,5\% para el par de  partidores que amplifican para IL-12 a 58ºC (Figura ref\{fig:il12\}), 115\% para el el par de partidores que amplifican para TNF-\$alpha\$ a 58ºC (Figura ref\{fig:tnfa\}), 120,9\% para el par de partidores que amplifican para IFN-\$gamma\$ a 61,5ºC (Figura ref\{fig:ifng\}), 105,9\% para el par de partidores que amplifican para IL-1\$beta\$ a 58ºC (Figure ref\{fig:il1b\}) y finalmente 102\% para la pareja de partidores que amplifican para iNOS a 58ºC (Figura ref\{fig:inos\}).

Todas las eficiencias son similares, la unica que se escapa un poco del promedio es la eficiencia del par de partidores que amplifican para IFN-\$gamma\$, esto puede deberse a que el producto o amplicón que producen es muy pequeño (\textasciitilde{}51pb) (Tabla ref\{tabla:partidores\}) y está en el limite de lo recomendado para la cuantificación por el metodo \$DeltaDelta C\_T\$ [@Bustin2009; @Pfaffl2001].

Teniendo ya estandarizados todos los partidores se procedió a evaluar las muestras biológicas del ensayo, en las cuales la tendencia demostró que la respuesta inmune empieza a aumentar pasado el día 14, ya que todos los peak de expresión se observaron en los días 21 y 28 según corresponda. (Figura ref\{fig:qpcr\}), esto se condice con varios estudios donde los tiempos de respuesta frente a \$beta\$-glucanos en tratamientos emph\{in vivo\} rondan dentro o después de los 21 días [@Morales-Lange2014; @Dobsikova2013; @Skov2012; @Casadei2012].

Para el caso de varios controles en distintas moleculas también se observo un aumento pasado este día, esto puede deberse a un estrés en los peces, el cual haya gatillado un aumento en la respuesta inmune como se ha demostrado en estudios anteriores [@Bricknell2005; @Magnadottir2006; @Barandica2008; @Bowden2008], pero aún teniendo controles altos en los ensayos de transcripción, por ejemplo para TNF-\$alpha\$ (Figura ref\{fig:qpcr\}A), IFN-\$gamma\$ (Figura ref\{fig:qpcr\}B) e IL-1\$beta\$ (Figura ref\{fig:qpcr\}C) estas diferencias entre tratamientos siguen siendo estadísticamente significativas (\$p $<$ 0,05\$).

Teniendo estos datos en cuenta se puede inferir que el suplementar la alimentación de los peces con emph\{Zimosán A\} liberado en dieta promoveria la expresión de los genes que codifican para distintas citoquinas pro-inflamatorias y moleculas efectoras de inmunidad.

section\{Objetivo 3\}

Para cuantificar las proteínas extraidas se utilizó el metodo BCA, en el cual se obtuvo como resultado una curva de calibrado con un coeficiente \$R\textasciicircum{}2 = 0,9976\$ lo que indica que la recta es lineal y las interpolaciones fueron válidas, esto así generando una cuantificación de proteínas estable en todas las muestras (Tabla ref\{tablaPROTEINAS\}), demostrando a su vez la estandarización previa de los metodos usados en el Laboratorio.

Los anticuerpos fueron a su vez validados usando ELISA indirecto, obteniendo distintas curvas evaluando su reacción con su antígeno correspondiente [@Mason1980].

Para los 5 anticuerpos usados en el estudio se obtuvieron curvas logaritmicas con un coeficiente \$R\textasciicircum{}2 $<$ 0,97\$, esto significa que a mayor concentración de inmunógeno (peptido sintético) el anticuerpo se va saturando, mientras que al inicio de la curva hay una linealidad en la reacción. Con estos resultados se aprobó el uso de estos anticuerpos en ensayos de ELISA indirectos con muestra biológica como antígeno.

Todas las moleculas en estudio aumentaron su biodisponibilidad con respecto a sus controles. Tomando en cuenta el dogma de la biología molecular debiese haber obtenido los peak de biodisponibilidad de Proteínas posteriormente a los de transcrito, y hubieron casos, en que hubo peaks al mismo día que en lo visto por PCR en tiempo real(Figuras ref\{fig:elisa\} y ref\{fig:qpcr\}). Esto se puede deber a que exista un intervalo de tiempo anterior al medido en que se pueda apreciar la diferencia entre ambas condiciones y la biodisponibilidad de proteínas que estamos observando corresponde a una traducción de transcrito de algun día anterior no evaluado, y finalmente, otra razón de este fenómeno sería la documentada presencia de leucocitos circulantes en las branquias [@Castro2014] los cuales estarían produciendo estas distintas moléculas reguladoras y efectoras de inmunidad.

Cabe destacar que la baja absorbancia obtenida en los ensayos se debe a que el efecto del Zimosán A a esa concentración produce solo un leve aumento en la respuesta inmune, lo cual está diseñado de esa forma, ya que con este estudio tampoco se espera que haya un estallido inflamatorio a nivel sistémico en el pez.

Sin embargo, a pesar de lo anteriormente mencionado, en los muestreos posteriores al día 14 se aprecia un aumento notable en la biodisponibilidad de todas las moléculas, con diferencias significativas frente a sus controles, lo que corrobora lo visto a nivel de transcrito, la liberación de Zimosán A en dieta genera una respuesta inmune detectable a nivel de mRNA y proteínas.

Los resultados planteados en esta tesis sentarían las bases para plantear que el receptor de \$beta\$-glucanos descrito para emph\{Salmo salar\} [@Guselle2006; @Morales-Lange2014] podría estar conservado dentro de la familia de los Salmonidos.

\end{document}
