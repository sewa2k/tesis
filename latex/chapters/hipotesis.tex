\chapter{Hipótesis}

La administración oral del $\beta$-glucano zimosán en _O.mykiss_ genera una respuesta inmune cuantificable en tejido branquial lo que permite establecer un modelo de la expresión de moléculas reguladoras y efectoras de la inmunidad.

\chapter{Objetivo General}

Establecer un modelo molecular basado en la cuantificación de diferentes parámetros de respuesta inmunológica expresada en tejido branquial _O.mykiss_ y relacionada con la administración oral del $\beta$-glucano zimosán.

\section{Objetivos Específicos}

1. Implementar un sistema de alimentación que permita realizar la administración oral de zimosán y sus respectivos controles a _O.mykiss_
2. Evaluar la expresión de moléculas efectoras y reguladoras de respuesta inmune en tejido branquial de _O.mykiss_ tratados con zimosán vía oral
3. Detectar la disponibilidad de proteínas efectoras y reguladoras de respuesta inmune en tejido branquial de _O.mykiss_ tratados con zimosán vía oral
4. Relacionar el nivel de expresión y detección de moléculas de respuesta inmune evaluadas en tejido branquial con el nivel de dosificación oral de zimosán.

\clearpage

